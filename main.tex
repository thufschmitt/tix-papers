\documentclass{article}

\usepackage[margin=1in]{geometry}
\usepackage{fontspec}
\usepackage[usenames,dvipsnames,svgnames,table]{xcolor}
\usepackage{amssymb}
\usepackage{amsmath}
\usepackage{bm}
\usepackage{stmaryrd}

\usepackage{mathtools}
\usepackage{mathpartir}
\newcommand{\irlabel}[1]{\text{\emph{(#1)}}}

\usepackage{xfrac}

\usepackage{listings}
\lstset{%
  escapeinside={//*}{*//}
  }

\usepackage{hyperref}
\usepackage{todo}

\date{}

\usepackage{syntax}
\renewcommand{\grammarlabel}[2]{\meta{#1 #2}}
\newcommand{\meta}[1]{\ensuremath{#1}} % For meta syntax
\renewcommand{\|}{\textrm{|}}
\def\a/{\meta{a}}
\def\b/{\meta{b}}
\def\c/{\meta{c}}
\def\e/{\meta{e}}
\def\E/{\meta{E}}
\def\f/{\meta{f}}
\def\o/{\meta{o}}
\def\p/{\meta{p}}
\def\q/{\meta{q}}
\def\r/{\meta{r}}
\def\s/{\meta{s}}
\def\t/{\meta{t}}
\def\u/{\meta{u}}
\def\v/{\meta{v}}
\def\x/{\meta{x}}

\newcommand{\assign}[2]{\ensuremath{\sfrac{#2}{#1}}}
\newcommand{\assignp} [2] {\assign{#1}{#2}}
\newcommand{\subst} [3] {#3 [\assign{#1}{#2}]}
\newcommand{\substp} [3] {#3 [\assignp{#1}{#2}]}
\newcommand{\dstep} [2] {#1 \ensuremath{\rightarrow} #2}
\newcommand{\ndstep} [2] {#1 \ensuremath{\nrightarrow} #2}
\newcommand{\ndsteps} [2] {#1 \ensuremath{\nrightarrow^*} #2}
\newcommand{\dstepa} [3] {\dstep{#1}{&#2}~\emph{#3} \\}

\newcommand{\eqdef}[2]{#1 \ensuremath{\overset{\text{def}}{=}} #2}
\newcommand{\eqdefa}[3]{\eqdef{#1}{&#2} \emph{#3} \\}

\newcommand{\xone}{\ensuremath{x_1}}
\newcommand{\xn}{\ensuremath{x_n}}
\newcommand{\eone}{\ensuremath{e_1}}
\newcommand{\etwo}{\ensuremath{e_2}}
\newcommand{\en}{\ensuremath{e_n}}
% TODO: redefine to get the 0 and 1 of types
% (with double bar)
\newcommand{\zero}{0}
\newcommand{\one}{1}


\newcommand{\ty}[1]{\texttt{#1}}
\newcommand{\set}[1]{\ensuremath{\mathcal{#1}}}
\newcommand{\undef}{\nabla}
\newcommand{\quasiconst}[1]{\overset{#1}{\twoheadrightarrow}}
\DeclareMathOperator\dom{dom}
\DeclareMathOperator\deff{def}
\DeclareMathOperator\var{\mathcal{V}}
\newcommand{\orthsum}{\oplus^\bot}
\newcommand{\orthplus}{\diamond}
\newcommand{\subtype}{\leq}
\newcommand{\onerec}{\{ \textbf{..} \}}
\DeclareCollectionInstance{plainmath}{xfrac}{mathdefault}{math}
{%
  slash-symbol = \sslash{}
}
\newcommand{\ofTypeP}[2]{\UseCollection{xfrac}{plainmath}\sfrac{#1}{#2}}
\newcommand{\matchType}[1]{\Lbag #1 \Rbag}

\newcommand{\Γ}{\Gamma}
\newcommand{\τ}{\ensuremath{\tau}}
\newcommand{\σ}{\sigma}
\DeclareMathOperator\any{\textsc{Any}}
\DeclareMathOperator\grad{\star}
\DeclareMathOperator\Int{Int}
\DeclareMathOperator\Bool{Bool}
\newcommand{\λ}{\lambda}
\newcommand{\recleq}{\sqsubsetleq}
\newcommand{\discrete}[2]{\left\{ #1, .., #2 \right\}}

\newcommand{\pref}[1]{\ref{#1} at page~\pageref{#1}}



\title{A type system for nix}

\begin{document}
\maketitle

\tableofcontents

\pagebreak

In order to lighten the type system, we will not work on the full nix language
(presented in Section~\ref{sec:nix-grammar}), but on a simplified version
called nix-light that we also presents in this document.

We also present a compilation from nix to nix-light.

\section{The nix language}

\subsection{Grammar}
\label{sec:nix-grammar}
The grammar of nix is given at Figures~\pref{grammar::nix}
and~\pref{grammar::types}. It consists\todo{Remove as soon as the grammar is
extended} of a simple lambda calculus with lists, constants and some type
annotations.

The set of constants $c$ constists of strings $s$, integers $i$, booleans $b$
and a set of builtins functions $f$.

\begin{figure}
  \begin{grammar}
    \bfseries
    <e> ::=
    $x$ \| $c$
    \alt $\λ$$p$.$e$ \| $e$ $e$
    \alt let $x$ = $e$; $\cdots{}$; $x$ = $e$; in $e$
    \alt [ $e$ $\cdots$ $e$ ]
    \alt if $e$ then $e$ else $e$

    <c> ::= $s$ \| $i$ \| $b$ \| $f$

    <p> ::= $q$ \| $q$@$x$ \| $r$

    <q> ::= ($r$, $r$) \| $q$:\τ

    <r> ::= $x$ | $x$:\τ
  \end{grammar}
  \caption{\label{grammar::nix}The nix grammar for expressions}
\end{figure}

\begin{figure}
  \begin{grammar}
  \bfseries
  <t> ::= $c$ \| $t$ $\bm{\rightarrow}$ $t$
    \alt $t$ $\bm{\vee}$ $t$ \| $t$ $\bm{\wedge}$ $t$ \| $t$ $\bm{\backslash}$ $t$
    \alt [\meta{R}]
    \alt bool \| int \| string

  <u> ::= $t$ \| ?$t$

  <R> ::= $t$ \| \meta{R^{\bm{+}}} \| \meta{R}* \| \meta{R}?
    \| \meta{R} \meta{R} \| \meta{R}\texttt{|}\meta{R}

  <\τ> ::= $t$ % No polymorphism for now
\end{grammar}

  \caption{\label{grammar::types}The nix and nix-light grammar for types}
\end{figure}


\subsection{Typing}
\todo{Write}

\subsection{Preprocessing}
\todo{Reformulate to fit the position in the document}
Although the nix language isn't a very complex one, it still contains a lot of
syntactic sugar that we want to get rid of, and because of its flexibility is
really hard to type − because a lot of patterns that are syntactic in other
languages are only determined by the semantics, and thus impossible to detect
statically.

In particular, most statically typed languages which have a notion of type at
runtime have a special syntactic construct to do case analysis on the type of
a variable. For example, in the CDuce language~\cite{Fri04}, this case
analysis is expressed by a special form of pattern that can be used in
pattern-matching. This is important because it means that those informations
can be used by the type-checker. In Nix, however, this case analysis is done by
the combination of if-then-else's constructs and functions that tell wether
their argument is of a certain type (\lstinline{isInt} which returns
\lstinline{true} if its argument is an integer for example).
The problem is that some expressions that are reasonable can't be typed without
enriching the typing environment in one of the branches of the if. For example,
one may expect the expression \lstinline{if isInt $x$ then $x + 1$ else $x$} to
be well typed, but this requires the type-checker to be aware that in the first
branch $x$ is of type \lstinline{Int}.

To work around this problem, we restrict ourselves to some syntactic patterns
that we can recognise. In practice, to keep the type system simple, we only try
to enrich the environment in presence of an expression of the form
\lstinline{if isT $x$ then $e_1$ else $e_2$}.

As this is rather limiting, we allow ourselves a pre-processing phase that
converts some slightly more complicated expressions into the pattern above −
while preserving the semantics of course.
The list of those conversions may be expanded at will − that's why it is kept
out of the type system, but the figure~\pref{semantics::pre-processing}
provides a few of them. (Although we did not give any formal semantics for nix
yet, we can at least say that those intuitively preserve the expected semantics
of a if-then-else construct).

\begin{figure}
  \begin{align}
    \text{if $e'_1$ || $e'_2$ then $e_1$ else $e_2$} &\rightarrow
      \text{if $e'_1$ then $e_1$ else (if $e'_2$ then $e_1$ else $e_2$)} \\
    \text{if $e'_1$ \&\& $e'_2$ then $e_1$ else $e_2$} &\rightarrow
      \text{if $e'_1$ then (if $e'_2$ then $e_1$ else $e_2)$ else $e_2$} \\
    \text{if not $e$ then $e_1$ else $e_2$} &\rightarrow
      \text{if $e$ then $e_2$ else $e_1$}
  \end{align}
  \caption{Simple pre-processing phase for nix expressions}\label{semantics::pre-processing}
\end{figure}


\section{Nix-light}

\subsection{Grammar}
\label{sec:nix-light-grammar}
\documentclass{article}

\usepackage{fontspec}
\usepackage[usenames,dvipsnames,svgnames,table]{xcolor}

\usepackage{syntax}

\title{Nix simplified grammar}
\date{}

\renewcommand{\grammarlabel}[2]{#1 #2}

\begin{document}

\maketitle{}

\begin{grammar}
<e> ::=
    x | c
  \alt e.a | e.a or e
  \alt p:e | e e
  \alt [ e ... e ] | \{ e = e; ...; e = e; \}
  \alt with e; e
  \alt if e then e else e
  \alt let x = e; ...; x = e; in e
  \textcolor{blue}{\alt e :: e | nil}

<a> ::= b. ... .b

<b> ::= \$\{e\} | l

<c> ::= s | i | ...

<p> ::= q | q@x | x

<q> ::= \{ f, ..., f \} | \{ f, ..., f, … \}

<f> ::= x | x ? e

\end{grammar}
\end{document}


\subsection{Semantics}
\begin{figure}
  \center
  \def\leadsto{\ensuremath{\rightsquigarrow}}
  \begin{tabular}{rl}
  °(λx.e1) e2° &\leadsto °e1[x := e2]° \\
  °(λp.e) v° &\leadsto °e[$\sfrac{p}{v}$]° \\
  °(x = v tin T) ? e1 : e2° &\leadsto °e1[x := v]° \quad if $\vdash v : T$ \\
  °(x = v tin T) ? e1 : e2° &\leadsto °e2[x := v]° \quad if $\vdash v : \lnot T$ \\
  °{ x = e; $\cdots$ }.x° &\leadsto °e° \\
  °{ x = e; $\cdots$ }.x or e'° &\leadsto °e° \\
  °{ x1 = e1; $\cdots$; xn = en }.x or e'° &\leadsto °e'°
      \quad if $x \notin \left\{ x1, \cdots, xn \right\}$ \\
  °e : τ° &\leadsto °e°
  \end{tabular}
  \caption{Sémantique opérationnelle de Nix-light\label{nix-light::semantics}}
\end{figure}


\section{Nix-light typing}

\subsection{The $\λ\&-calculus$}
\begin{mathpar}
  \inferrule{ }{\Γ \tinfer x:\Γ(x)}(IVar)
  \and\inferrule{ \Γ(x) \subtypeG \τ }{\Γ; \tcheck x:\τ}(CVar)

  \and\inferrule{ }{\Γ \tinfer c:\Bt(c)}(IConst)
  \and\inferrule{ \Bt(c) \subtypeG \τ }{\Γ \tcheck c:\τ}(CConst)

  \and\inferrule{%
    \Γ \tinfer e_1 : \τ_1 \\ \Γ \tinfer e_2 : \τ_2 \\
    \τ_1 \subtypeG \zero \rightarrow \one \\
    \τ_2 \subtypeG \dom(\τ_1)
  }{%
    \Γ \tinfer e_1~e_2 : \τ_1 \circ \τ_2
  }
  (IApp)

  \and\inferrule{%
    \Γ \tinfer e_2 : \σ \\
    \Γ \tcheck e_1 : \σ \rightarrow \τ
  }{%
    \Γ \tcheck e_1~e_2 : \τ
  }
  (CApp)

  \and\inferrule{%
    \Γ' \dashv p:\accept{p} \\ % Replace by the type accepted by p
    \Γ;\Γ' \tinfer e : \τ
  }{%
    \Γ \tinfer \λ p.e : \accept{p} \rightarrow \τ
  }
  (IAbs)

  \and\inferrule{%
    \τ \subtype \zero \rightarrow \one \\
    \forall \σ_1 \rightarrow \σ_2 \in \A(\τ), \\
      \Γ' \dashv p:\σ_1 \\ \Γ;\Γ' \tcheck e : \σ_2
  }{%
    \Γ \tcheck \λ p.e : \τ
  }(CAbs)

  \and
  \inferrule{%
    \forall i \in \discrete{1}{n},
      \Γ; x_1 : ?; \cdots; x_n : ? \tinfer e_i : \τ_i \\
    \Γ; x_1 : \τ_1; \cdots; x_n : \τ_n \tIC e : \τ
  }{%
    \Gamma \tIC \text{let } x_1 = e_1; \cdots; x_n = e_n
      \text{ in } e : \τ
  }
  (Let)

  \and
  \inferrule{%
    \forall i \in \discrete{1}{n},
      \Γ; x_1 : \τ_1; \ldots; x_n : \τ_n \tcheck e_i : \τ_i \\
    \Γ; x_1 : \τ_1; \cdots; x_n : \τ_n \tIC e : \τ
  }{%
    \Gamma \tIC \text{let } x_1 : \τ_1 = e_1; \ldots{}; x_n : \τ_n = e_n
      \text{ in } e_0 : \τ_0
  }
  (LetAnnot)

  \and\inferrule{%
    \Γ \tinfer e : \τ \\
    \τ \not\subtype t \Rightarrow \Γ; x : \τ \wedge \lnot t \tinfer e_2 : \τ_2 \\
    \τ \not\subtype \lnot t \Rightarrow \Γ; x : \τ \wedge t \tinfer e_1 : \τ_1 \\
  }{%
    \Γ \tinfer (x = e \in t) ? e_1 : e_2 : \τ_1 \vee \τ_2
  }(ITcase)

  \and\inferrule{%
    \Γ \tinfer e : \τ \\
    \τ \not\subtype t \Rightarrow \Γ; x : \τ \wedge \lnot t \tcheck e_2 : \σ \\
    \τ \not\subtype \lnot t \Rightarrow \Γ; x : \τ \wedge t \tcheck e_1 : \σ \\
  }{%
    \Γ \tcheck (x = e \in t) ? e_1 : e_2 : \σ
  }(CTcase)
\end{mathpar}


\subsection{Bidirectional typing}
\subsubsection{Motivation and overview}
\label{motivation-and-overview}

The rules defined above are already quite expressive. However, they aren't
enough to type the following function, no matter how the type annotations are
written:

\begin{lstlisting}[language=NLight]
  (lambda cond. lambda x.  (y := cond in true) ? x+1 : not x
  : (true -> Int -> Int & false -> Bool -> Bool))
\end{lstlisting}

Indeed, do to this we would need to annotate $x$ as \ty{Int} or \ty{Bool}
depending on wether \texttt{cond} was \ty{true} or \ty{false}, which isn't
possible.

To remedy this problem, we split our type system in two parts: an inference
part and a checking part.  The inference part − denoted with typing judgements
of the form $\Γ \tinfer e : \τ$ − corresponds to classical bottom-up
type-inference, while the checking part − denoted with typing judgements of the
form $\Γ \tcheck e : \τ$ corresponds to a top-down type inference, where the
type of the expression is already known and we use it to infer the type of the
sub-expressions − in other words, we propagate the type annotations to the
bottom while type-checking. In particular, this ``checking'' type-system allows
us a more precise typing of lambdas.

Explicitly annotated let-bindings are thus typed using the checking type-system
and we can rewrite the previous expression as
\begin{lstlisting}[language=NLight]
let f : (true -> Int -> Int & false -> Bool -> Bool) =
 lambda cond. lambda x.  (y := cond tin true) ? x+1 : not x
in f
\end{lstlisting}

In this case, we just have to \emph{check} that
\begin{lstlisting}[language=NLight]
lambda x. (y := cond tin true) ? x+1 : not x
\end{lstlisting}
has type $\ty{Int} \rightarrow \ty{Int}$ under the hypothesis $\texttt{cond} :
\ty{true}$ and $\ty{Bool} \rightarrow \ty{Bool}$ under the hypothesis
$\texttt{cond} : \ty{false}$. This means checking that
\begin{lstlisting}[language=NLight]
(y := cond tin true) ? x+1 : not x
\end{lstlisting}
has type $\ty{Int}$ under the hypothesis $\texttt{cond} : \ty{true}; \texttt{x}
: \ty{Int}$ and $\ty{Bool}$ under the hypothesis $\texttt{cond} : \ty{false};
\texttt{x} : Bool$. This is true, thanks to the ruls CTcase of the type system
(figure~\pref{typing::lambda-calculus}).

So the basic idea behind this is to use the type annotations of parent nodes
(or any other type information that we got from upper in the AST), not just to
check that the inferred type was correct, but in the inference process itself.

\subsubsection{Pairs}\todo{Replace by lists}

For example, if we want to type the expression $(e_1, e_2) : (\τ_1 \times
\τ_2)$, we will try to type $(e_1 : \τ_1)$ and $(e_1 : \τ_2)$ and then merge
the results.
In a more general fashion, if we got $(e_1, e_2) : \τ$ − where $\τ$ is any
subtype of $\one \times \one$ − we will need to project $\τ$ on its first and
second component respectively to type $e_1$ and $e_2$.

We thus extends this definition of the projections $\pi_1$ and $\pi_2$ to more
complex pair types (\emph{ie.} subtypes of $\one \times \one$), keeping the
invariant that for a given subtype $\τ$ of $\one \times \one$, $\pi_1(\τ)
\times \pi_2(\τ)$ is a subtype~\todo{Gradual ?} of $\τ$, so that if $e_1$ has
type $\pi_1(\τ)$ and $e_2$ has type $\pi_2(\τ)$, then $(e_1, e_2)$ can be given
the type $\τ$.

Let $\τ$ be a subtype of $\one \times \one$.
We know that $\τ$ can be written as:
\[
  \τ = \bigvee\limits_{i\in I}\left(\bigwedge\limits_{p\in P_i} (\σ_p \times \τ_p)
  \wedge \bigwedge\limits_{n \in N_i} \lnot (\σ_n \times \τ_n) \right)
\]
We define $\pi_1(\τ)$ (resp. $\pi_2(\τ)$) as
\begin{align*}
  \pi_1(\τ) &= \bigwedge\limits_{i \in I} \bigwedge\limits_{N' \subseteq N}
    \left(\bigwedge\limits_{p \in P_i} \σ_p \wedge
    \bigwedge\limits_{n \in N'} \lnot \σ_n \right)\\
  \pi_2(\τ) &= \bigwedge\limits_{i \in I} \bigwedge\limits_{N' \subseteq N}
    \left(\bigwedge\limits_{p \in P_i} \τ_p \wedge
    \bigwedge\limits_{n \in N'} \lnot \τ_n \right)
\end{align*}

We can check\todo{Check that this is indeed true} that this definition respects
our constraint.

\subsubsection{Arrows}

Another construct for which we want to propagate type informations is the
definition of a function.

Assume we got an expression $(\λ p . e) : \τ$ where $\τ$ is a subtype (not
gradual) of $\zero \rightarrow \one$.

We want to type the function for each concrete arrow type included in $\τ$. In
other words, if we note $\A(\τ)$ the set of all arrow types in $\τ$, we want that
for all $\σ \rightarrow \σ' \in \A(\τ)$, $p$ matches $\σ$, and that under this
matching, $e$ has type $\σ'$.

This is given by the rule \emph{CAbs} of the type system in the
Figure~\pref{typing::lambda-calculus}.

Remains the definition of $\A(\τ)$, that we give as follows:

If $\τ$ is in the form
\[
  \τ = \bigvee\limits_{i\in I}\left(
    \bigwedge\limits_{p\in P_i} (\σ_p \rightarrow \τ_p)
    \wedge \bigwedge\limits_{n \in N_i} \lnot (\σ_n \rightarrow \τ_n)
  \right)
\]
then $\A(\τ)$ is given by:
\[
  \A(\τ) = \bigsqcup\limits_{i \in I} \{ \σ_p \rightarrow \τ_p \| p \in P_i \}
\]
where $\sqcup$ is defined as
\[
  \{ \σ_i \rightarrow \τ_i \| i \in I \} \sqcup \{ \σ_j \rightarrow \τ_j \| j \in J \} =
    \{ (\σ_i \wedge \σ_j) \rightarrow (\τ_i \vee \τ_j) \| i \in I, j \in J \}
\]

In the example of the Section~\ref{motivation-and-overview}, the expression has
type $t = (true \rightarrow (Int \rightarrow Int)) \wedge (false \rightarrow
(Bool \rightarrow Bool))$

Thus, we have $\A(t)$ equal to the set $\left\{ true \rightarrow (Int
\rightarrow Int); false \rightarrow (Bool \rightarrow Bool) \right\}$


\subsection{Soundness}
We have the two classical results of \emph{Subject reduction} and
\emph{Progress} which entail type soundess for the non-gradual part of the
language.
Note that because we defined two typing judgements, both result will be stated
twice.

We first prove the following lemmas:

\begin{lemma}\label{lemma:inferCheck}
  Let $e$ be an expression, $\Γ$ a typing environment and $\τ$ and $\τ'$ two
  (possibly gradual) types with $\τ \subtypeG \τ'$.
  If $\Γ \tinfer e : \τ$ then $\Γ \tcheck e : \τ'$.
\end{lemma}

\begin{proof}
  \todo{}
\end{proof}

\begin{lemma}[Substitution]\label{lemma:substitution}
  Let $e$ and $e'$ be expressions, $x$ be a variable, $\τ$ and $\τ'$ two types,
  $\Γ$ a typing environment and $\delta$ be $\Uparrow$ or $\Downarrow$.

  If $\Γ; x : \τ' \tIC e : \τ$ and $\Γ \tinfer e' : \τ'$ then $\Γ \tIC
  \subst{x}{e'}{e} : \τ$
\end{lemma}

\begin{proof}
  By induction on the typing derivation of $\Γ; x : \τ' \tIC e : \τ$. We
  replace every Ivar rule introducing $\Γ ; x : \τ' \tinfer x : \τ'$ by a
  derivation of $\Γ \tinfer e' : \τ'$, and every Cvar rule introducing $\Γ ; x
  : \τ' \tcheck x : \τ''$ (with $ \τ' \subtypeG \τ'')$ by a derivation of $\Γ
  \tcheck e' : \τ''$ (which exists because of Lemma~\ref{lemma:inferCheck}).

  This builds a new derivation of $\Γ \tinfer \subst{x}{e'}{e} : \τ$.
\end{proof}

\begin{theorem}[Subject reduction $\Uparrow$]
  For any pair $e, e'$ of terms (of nix-light), if $\Γ \tinfer e : t$ and $e
  \rightarrow e'$, then $\Γ \tinfer e' : t$.
\end{theorem}

\begin{proof}
  We consider an expression $e$ such that $\Γ \tinfer e : t$.
  We prove by induction on the derivation of $\Γ \tinfer e : t$ that $\forall
  e', (e \rightarrow e') \Rightarrow (\Γ \tinfer e' : t)$.

  Let's consider the various possibilities for the last rule of the derivation
  $\Γ \tinfer e : t$.

  \begin{description}
    \item[IVar,IConst,IAbs,ICons] The expression $e$ is a value and can't be
      reduced, so the property holds.
    \item[IApp]
      $\inferrule{%
        \Γ \tinfer e_1 : t_1 \\ \Γ \tinfer e_2 : t_2 \\
        t_1 \subtypeG \zero \rightarrow \one \\
        t_2 \subtypeG \dom(t_1)
      }{%
        \Γ \tinfer e_1~e_2 : t_1 \circ t_2
      }$

      The expression $e$ has then the form $e_1~e_2$ with $\Γ \tinfer e_1
      : t_1$ and $\Γ \tinfer e_2 : t_2$ (and $t = t_1 \circ t_2$).
      It can be reduced in three different ways (depending of the form of $e_1$
      and $e_2$):
      \begin{itemize}
        \item If $e_1$ is a value $\λ r. e_0$, then the only possible reduction
          is by applying the $\beta$-reduction rule, so the only choice for
          $e'$ is $\subst{x}{e_2}{e_0}$ (where $x = \var(r)$).

          Moreover, a case analysis on the different typing rules shows that
          the last rule of the derivation of $\Γ \tinfer \λ r . e_0 : t_1$ can
          only be the Iabs rule:
          \[
            \inferrule{%
              x:t_x \vdash r:t_x \\ \Γ; x:t_x \tinfer e_0:s
            }{%
              \Γ \tinfer \λ r.e_0 : t_x \rightarrow s
            }
          \]
          So $t_1$ has the form $t_x \rightarrow s$. This means that $t_2$ is a
          subtype of $t_x = \dom(t_x \rightarrow s)$ and that $t'$ is equal to
          $s = (t_x \rightarrow s) \circ t_2$.

          Moreover, as $\Γ; x:t_x \tinfer e_0:s$ and $\Γ \tinfer e_2 : t_2$
          with $t_2 \subtypeG t_x$, the Lemma~\ref{lemma:substitution} allows
          us to conclude that $\Γ \tinfer e' : t'$.
        \item If $e_1$ is a value $\λ p.e_0$ (with $p$ not in the form $x$ or
          $x:t$), and $e_2$ is a value, then we can use the same reasoning.
        \item In the other cases, we can reduce either $e_1$, either $e_2$.

          If we reduce $e_1$ to $e'_1$, we get a new expression $e' =
          e'_1~e_2$, and $e'_1$ satisfies $\Γ \tinfer e'_1 : t_1$ (by induction
          hypothesis). By re-applying the Iapp rule, we get $\Γ \tinfer
          e'_1~e_2 : t$, thus $\Γ \tinfer e' : t$.

          The same holds if we reduce $e_2$.
      \end{itemize}
  \end{description}
\end{proof}


\section{Compilation}

\subsection{Compilation rules}
\documentclass[frenchb]{scrartcl}
% \usepackage[margin=1in]{geometry}

% Put some french in it
\usepackage{babel}

\usepackage[toc,page]{appendix}

% And hypertext links everywhere
\usepackage{hyperref}

% To get a proper text encoding with xelatex
\usepackage{fontspec}
\setmainfont{Crimson-Roman.otf}[
  Path =./fonts/,
  BoldFont = Crimson-Bold.otf,
  ItalicFont = Crimson-Italic.otf,
  BoldItalicFont = Crimson-BoldItalic.otf
]

% Render inference rules
\usepackage{mathpartir}
% The labels of those rules
\newcommand{\irlabel}[1]{\text{\emph{(#1)}}}

% Fore code blocks
\usepackage{listings}
\lstdefinelanguage{NLight}{%
  morekeywords={%
    let,in,if,then,else,Cons,Nil
  },%
  morekeywords={[2]Int,true,false,Bool,T},   % types go here
  otherkeywords={:,=[]}, % operators go here
  literate={% replace strings with symbols
    {->}{{$\to$}}{2}
    {lambda}{{$\lambda$}}{1}
    {tin}{{$\in$}}{2}
    {\&}{{$\wedge$}}{2}
  },
  basicstyle={\sffamily},
  keywordstyle={\bfseries},
  keywordstyle={[2]\itshape}, % style for types
  keepspaces,
  mathescape % optional
}[keywords,comments,strings]
\lstset{%
  escapeinside={//*}{*//},
  breaklines=true,
  mathescape=true,
  language=NLight
}

\title{Rapport de stage}
\author{Théophane Hufschmitt}
\date{21 Août 2017}

\begin{document}
\maketitle

\tableofcontents

\pagebreak

\section*{Abstract}

\section{Contexte}
% État de l'art, motivation du stage

\subsection{Le langage Nix}
% Description de Nix et de toutes les horreurs qu'il contient
% Explication rapide de ce qui est nécessaire pour le typer à peu près
% raisonnablement

\subsection{CDuce}
% Description du système de type de CDuce et de pourquoi il peut être en grande
% partie réutilisé

\section{Nix-light} % TODO: find another name for this

\subsection{Motivation}
% Explication de pourquoi nix est trop permissif et pourquoi il vaut mieux
% bosser sur autre chose.

\subsection{Description}
% Description du langage, grammaire + sémantique

\section{Typage}

\subsection{Types}
% Présentation des types utilisés

\subsubsection{Syntaxe}
\subsubsection{Sous-typage}
% Discussion autour du sous-typage lazy
% Sous-typage graduel

\subsection{Lambda-calcul}
% Typage du langage sans records et sans listes

\subsection{Structures de données}
% Description du typage des deux structures de données de Nix

\subsubsection{Extension des types}
% Ajout des types Cons et record.

\subsubsection{Listes}
% Typage des listes. Rien de très compliqué, mais les regexp-lists nécessitent
% peut-être un peu d'explication. À voir si on garde comme une sous-partie ou
% si on merge dans la section "lambda-calcul", vu que c'est ni central ni
% original (mais joli par contre)

\subsubsection{Records}
% Typage des records. Probablement plein de choses à dire ici.

\section{Soundness du typage}
% Difficulté de définir la soundness avec le type graduel
% Blablater sur la difficulté des preuves.

\section{Implémentation}
% Tout ce qui concerne l'implémentation. Probablement des choses à dire

\begin{appendices}
  \section{Preuves de typage}
\end{appendices}

\bibliographystyle{alpha}
\bibliography{references}
\end{document}


\subsection{Preservation of typing}

\todos{}

\bibliographystyle{alpha}
\bibliography{references}
\end{document}
