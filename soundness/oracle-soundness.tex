We now prove the classical subject-reduction and progress for the oracle
type-system.

We first prove the following lemmas:

\begin{lemma}[Admissibility of the sumbsumption]
  In the oracle type-system, the sumbsumption rule stated as
  \[
    \inferrule{\Γ \toracle e : t\\ t \subtype t'}{\Γ \toracle e : t'}
  \]
  is admissible.
\end{lemma}

\begin{proof}
  \todo{}
\end{proof}

\begin{lemma}[Substitution]\label{lemma:substitution}
  Let $e$ and $e'$ be expressions, $x$ be a variable, $t$ and $t'$ two types and
  $\Γ$ a typing environment.

  If $\Γ; x : t' \toracle e : t$ and $\Γ \toracle e' : t'$ then $\Γ \toracle
  \subst{x}{e'}{e} : t$
\end{lemma}

\begin{proof}
  By induction on the typing derivation of $\Γ; x : t' \toracle e : t$. We
  replace every Ovar rule introducing $\Γ ; x : t' \toracle x : t''$ by a
  derivation of $\Γ \toracle e' : t''$.

  This builds a new derivation of $\Γ \toracle \subst{x}{e'}{e} : t$.
\end{proof}

\begin{theorem}[Subject reduction]\label{thm:subj-reduction-oracle}
    For any pair $e, e'$ of terms (of nix-light), if $\Γ \toracle e : t$ and $e
      \rightarrow e'$, then $\Γ \toracle e' : t$.
\end{theorem}

\begin{proof}
  We consider an expression $e$ such that $\Γ \toracle e : t$.
  We prove by induction on the derivation of $\Γ \toracle e : t$ that $\forall
  e', (e \rightarrow e') \Rightarrow (\Γ \toracle e' : t)$.

  Let's consider the various possibilities for the last rule of the derivation
  $\Γ \toracle e : t$.

  \begin{description}
    \item[OVar,OConst,OAbs,OCons] The expression $e$ can't be
      reduced, so the property holds.
    \item[OApp]
      $\inferrule{%
        \Γ \toracle e_2 : s \\
        \Γ \toracle e_1 : s \rightarrow t
      }{%
        \Γ \toracle e_1~e_2 : t
      }$

      The expression $e$ has then the form $e_1~e_2$ with $\Γ \toracle e_1
      : s \rightarrow t$ and $\Γ \toracle e_2 : s$.
      It can be reduced in three different ways (depending of the form of $e_1$
      and $e_2$):
      \begin{itemize}
        \item If $e_1$ is a value $\λ r. e_0$, then the only possible reduction
          is by applying the $\beta$-reduction rule, so the only choice for
          $e'$ is $\subst{x}{e_2}{e_0}$ (where $x = \var(r)$).

          Moreover, a case analysis on the different typing rules shows that
          the last rule of the derivation of $\Γ \toracle \λ r . e_0 : t_1$ can
          only be the \textbf{OAbs} rule:
          \[
            \inferrule{%
                x:s \dashv x:s \\ \Γ;x:s \toracle e : t
            }{%
              \Γ \toracle \λ p.e : s \rightarrow t
            }
          \]
          As $\Γ; x:s \toracle e_0:t$ and $\Γ \toracle e_2 : s$, the
          Lemma~\ref{lemma:substitution} gives us that $\Γ \toracle e' : t$.

        \item If $e_1$ is a value $\λ p.e_0$ (with $p$ not in the form $x$ or
          $x:t$), and $e_2$ is a value, then we can use the same reasoning.
        \item In the other cases, we can reduce either $e_1$, either $e_2$.

          If we reduce $e_1$ to $e'_1$, we get a new expression $e' =
          e'_1~e_2$, and $e'_1$ satisfies $\Γ \toracle e'_1 : s$ (by induction
          hypothesis). By re-applying the Iapp rule, we get $\Γ \toracle
          e'_1~e_2 : t$, thus $\Γ \toracle e' : t$.

          The same holds if we reduce $e_2$.

      \end{itemize}
      \item[Let]
        $\inferrule{%
          \Γ; x_1 : t_1 \toracle e_1 : t_1\\
          \Γ ; x_1 : t_1 \toracle e_2 : t
        }{%
          \Γ \toracle \text{let } e_1 = x_1 \text{ in } e_2 : t
        }$

        The expression $e$ has the form ``let $e_1 = x_1$ in $e_2$''.
        It reduces (and may only reduce to) $e' = \subst{x_1}{e_1}{e_2}$.
        By applying the substitiution lemma, we can deduce that $e'$ has type
        $t$.
      \item[LetAnnot]
        The same reasoning also holds for the \textbf{LetAnnot} rule.
      \item[OTcase]
        $\inferrule{%
          \Γ \toracle e_0 : t_0 \\
          t_0 \not\subtype s \Rightarrow \Γ; x : t_0 \wedge \lnot s \toracle e_2 : t \\
          t_0 \not\subtype \lnot s \Rightarrow \Γ; x : t_0 \wedge s \toracle e_1 : t \\
        }{%
          \Γ \toracle (x = e_0 \in s) ? e_1 : e_2 : t
        }$

        If $e_0$ is not a value, then the only possible reduction for $e$ is to
        reduce $e_0$ to an expression $e'_0$ which will have a type $t_0$ under
        the context $\Γ$ (by induction hypothesis).
        By re-applying the \textbf{OTcase} rule, we obtain that
        $\Γ \toracle (x = e'_0 \in s) ? e_1 : e_2 : t$

        If $e_0$ is a value $v$, then the only possible reductions are if
        $\toracle v : s$ or $\toracle v : \lnot s$.
        Assume the first one. The expression $e$ then reduces to
        $\subst{x}{v}{e_1}$, and (by application of the substitution Lemma), we
        know that $\Γ \toracle \subst{x}{v}{e_1} : t$, thus $\Γ \toracle e': t$.

        By symmetry, this also holds if $\toracle v : \lnot s$.
  \end{description}
\end{proof}
