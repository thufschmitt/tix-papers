The type-system is defined in~\autoref{fig::oracle-type-system}.

\begin{figure}
  \begin{mathpar}
    \and\inferrule{ \Γ(x) \subtype t }{\Γ; \toracle x:t}(OVar)

    \and\inferrule{ \Bt(c) \subtype t }{\Γ \toracle c:t}(OConst)

    \and\inferrule{%
      \Γ \toracle e_2 : s \\
      \Γ \toracle e_1 : s \rightarrow t
    }{%
      \Γ \toracle e_1~e_2 : t
    }
    (OApp)

    \and\inferrule{%
      t \subtype \zero \rightarrow \one \\
      \forall s_1 \rightarrow s_2 \in \A(t), \\
        \Γ' \dashv p:s_1 \\ \Γ;\Γ' \toracle e : s_2
    }{%
      \Γ \toracle \λ p.e : t
    }(OAbs)

    \and
    \inferrule{%
      \forall i \in \discrete{1}{n},
        \Γ; x_1 : t_i \toracle e_i : t_i \\
      \Γ; x_1 : t_1 \toracle e : t
    }{%
      \Gamma \toracle \text{let } x_1 = e_1 \text{ in } e : t
    }
    (OLet)

    \and
    \inferrule{%
      \forall i \in \discrete{1}{n},
        \Γ; x_1 : t_1; \ldots; x_n : t_n \toracle e_i : t_i \\
      \Γ; x_1 : t_1; \cdots; x_n : t_n \toracle e : t
    }{%
      \Gamma \toracle \text{let } x_1 : t_1 = e_1; \ldots{}; x_n : t_n = e_n
        \text{ in } e : t
    }
    (OLetAnnot)

    \and\inferrule{%
      \Γ \toracle e : t \\
      t \not\subtype t \Rightarrow \Γ; x : t \wedge \lnot t \toracle e_2 : s \\
      t \not\subtype \lnot t \Rightarrow \Γ; x : t \wedge t \toracle e_1 : s \\
    }{%
      \Γ \toracle (x = e \in t) ? e_1 : e_2 : s
    }(OTcase)

    \and\inferrule{%
      \Γ \toracle e_1 : t_1 \\ \Γ \toracle e_2 : t_2 \\
      t_2 \subtype \cons(\any, \any) \\
      \cons(t_1, t_2) \subtype t
    }{%
      \Γ \toracle \cons(e_1, e_2) : t
    }(OCons)
  \end{mathpar}
  \caption{The oracle type-system}\label{fig::oracle-type-system}
\end{figure}
