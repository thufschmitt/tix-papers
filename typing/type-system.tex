\subsection{The $\λ\&-calculus$}

\subsubsection{Patterns}

In order to type patterns, we need to introduce a new form of typing judgement.
The judgement $\Γ \dashv p:\τ$ means that when applied against a type $\τ$, the
pattern $p$ will enrich the environment with the constraints $\Γ$.

For example, we got $x:\Int \dashv x:\Int$, which reads ``If we apply a term
of type $\Int$ to the pattern $x$, then the environment on the
right of the pattern will be enriched with the constraint $x:\Int$''

Likewise, the following statement holds.
\[x:\Int; y: \Bool \dashv \left\{ x; y ? \text{true}; \right\} : \{ x = \Int; y =? \Bool \}\]
This means that when if we match a term of type $\left\{ x =\Int; y =? \Bool
\right\}$ against the pattern $\left\{ x; y ? \text{true}; \right\}$, then the
environment on the right side of the pattern will be enriched with the
constraints $x : \Int$ and $y : \Bool$.

As the symbol ``$\dashv$'' suggests, this typing judgement is the converse of
the classical typing judgement $\Γ \vdash e : \τ$ for expressions: instead of
stating that under the hypothesis $\Γ$, the expression $e$ has type $\τ$, we
state that if the pattern $p$ has type $\τ$, then in produces the environment
$\Γ$.

The typing rules for this statement are given by the
figure~\pref{typing::patterns::typing-rules}.

Maybe~\todo{Find out wether this is true} this enjoys principal typing.

\subsubsection{Typecase}

The typecase $(x := e \in t) ? e_1 : e_2$ can be typed in a simple way, by
saying that if $e$ has a type $τ$, $e_1$ has type $\τ_1$ and $e_2$ has type
$\τ_2$ (under the current typing environment $\Γ$), then $(x := e \in t) ? e_1
: e_2$ has type $\τ_1 \vee \τ_2$.
However, doing so means that we do not use the extra type information given by
"$e \in t$", which loosens a lot the interest of this construct. For example,
an expression such as $(x := e \in \bm{{Int}}) ? x + 1 : x$, with $\vdash e :
\any$ wouldn't typecheck, as $x+1$ isn't well typed without any further
constraint on the type of $x$.

A more interesting typing rule would state that if $\Γ; x:\τ \wedge t \vdash
e_1: \τ_1$ and $\Γ; x:\τ \wedge \lnot t \vdash e_2: \τ_2$ (where $\τ$ is a type
of $e$ under the hypothesis $\Γ$), then the whole expression has type $\τ_1
\vee \τ_2$.
With this rule, the expression $(x := e \in \bm{{Int}}) ? x + 1 : x$ is
well-typed (provided that $e$ is).

The typing rules are given by the
figures~\pref{typing::lambda-calculus},~\pref{typing::records}
and~\pref{typing::operators}.

\begin{figure}
  \begin{mathpar}
    \inferrule{~}{x:\τ \dashv x:\τ}
    \and\inferrule{~}{\dashv \text{nil} : \text{nil}}
    \and\inferrule{\Γ \dashv p:\τ}{\Γ \dashv (p:\τ):\τ}
    \and\inferrule{%
      \Γ \dashv p : \τ \\ \mathcal{B}(c) \subtype \τ
    }{%
      \Γ \dashv p ? c : \τ
    }
    \and\inferrule{%
      \forall i \in \discrete{1}{n}, \Γ_i \dashv r_i : \τ_i \\
      \forall i, j \in \discrete{1}{n}, \text{Vars}(\Γ_i) \cap \text{Vars}(\Γ_j) = \varnothing \\
      \forall i, j \in \discrete{1}{n}, x_i \neq x_j
    }{%
      \Γ_1; \cdots; \Γ_n \dashv \left\{ x_1 = r_1; \cdots; x_n = r_n; \right\}
        : \left\{ x_1 = \τ_1; \cdots; x_n = \τ_n; \right\}
    }
    \and\inferrule{%
      \forall i \in \discrete{1}{n}, \Γ_i \dashv r_i : \τ_i \\
      \forall i, j \in \discrete{1}{n}, \text{Vars}(\Γ_i) \cap \text{Vars}(\Γ_j) = \varnothing \\
      \forall i, j \in \discrete{1}{n}, x_i \neq x_j
    }{%
      \Γ_1; \cdots; \Γ_n \dashv \left\{ x_1 = r_1; \cdots; x_n = r_n; \ldots \right\}
        : \left\{ x_1 = \τ_1; \cdots; x_n = \τ_n; \ldots \right\}
    }
    \and\inferrule{%
      \Γ_1 \dashv p_1 : \τ_1 \\
      \Γ_2 \dashv p_2 : \τ_2 \\
      \text{Vars}(\Γ_1) \cap \text{Vars}(\Γ_2) = \varnothing
    }{%
      \Γ_1; \Γ_2 \dashv \text{Cons}(p_1, p_2) : \text{Cons}(\τ_1, \τ_2)
    }
    \and\inferrule{%
      \Γ \dashv p : \τ \\
      x \notin \Γ
    }{%
      \Γ; x:\τ \dashv p@x : \τ
    }
  \end{mathpar}
  \caption{Typing rules for the patterns\label{typing::patterns::typing-rules}}
\end{figure}

\begin{figure}
  \begin{mathpar}
  \inferrule{ }{\Gamma; x:\tau \vdash x:\tau}(Var)

  \and
  \inferrule{ }{\Gamma \vdash c:\mathcal{B}(c)}(Const)

  \and
  \inferrule{%
    \Gamma \vdash e_1 : \tau_1 \\ \Gamma \vdash e_2 : \tau_2 \\
    \tau_1 \subtype \zero \rightarrow \one \\
    \tau_2 \subtype \dom(\tau_1)
  }{%
    \Gamma \vdash e_1 e_2 : \tau_1 \tau_2
  }
  (App)

  \and
  \inferrule{%
    \forall i \in \discrete{1}{n}, \\
      \Gamma; \ofTypeP{p}{\sigma_i} \vdash e: \tau_i \\
  }{%
    \lambda \left(p:{\bigvee\limits_{i=1}^n\σ}\right).e :
      \bigwedge\limits_{i=1}^n \left(\σ \rightarrow \τ\right)
  }
  (Abs)

  \and
  \inferrule{%
    \Gamma \vdash e : \tau \\
      \Γ; x : \τ \wedge t \vdash e_1 : \sigma_1 \\
      \Γ; x : \τ \wedge \lnot t \vdash e_2 : \sigma_2
  }{%
    \Gamma \vdash ((x := e \in t) ? e_1 : e_2) : \sigma_1 \vee \sigma_2
  }
  (Tcase)

  \and
  \inferrule{%
    \forall i \in \discrete{1}{n}, \Γ \vdash e_i : \τ_i \\
      \Γ; x_1 : \τ_1; \ldots; x_n : \τ_n \vdash e : \τ
  }{%
    \Gamma \vdash \text{let } x_1 = e_1; \ldots{}; x_n = e_n
      \text{ in } e : \τ
  }
  (Let)

  \and
  \inferrule{%
    \forall i \in \discrete{0}{n},
      \Γ; x_1 : \τ_1; \ldots; x_n : \τ_n \vdash e_i : \τ_i
  }{%
    \Gamma \vdash \text{let rec } x_1 : \τ_1 = e_1; \ldots{}; x_n : \τ_n = e_n
      \text{ in } e_0 : \τ_0
  }
  (LetRec)
\end{mathpar}

  \caption{Typing rules for the $\λ\&-calculus$\label{typing::lambda-calculus}}
\end{figure}
\begin{figure}
    \begin{mathpar}
  \inferrule{%
    \Γ \vdash e' : \τ \\
  \Γ \vdash e : \bigvee\limits_{i=1}^n s_i
  }{%
    \Γ \vdash \left\{ e = e'; \right\} :
  \bigvee\limits_{i=1}^n \left\{ s_i: \τ \right\}
  }
  \irlabel{RFinite}
  \and
  \inferrule{%
    \Γ \vdash e' : \τ \\
  \Γ \vdash e : \σ \\
  \σ \subtype \ty{string}
  }{%
    \Γ \vdash \left\{ e = e'; \right\} :
  \left\{ \_: \τ \vee \undef \right\}
  }
  \irlabel{RInfinite}
  \and
  \inferrule{%
    \Γ \vdash e_1 : \tau \\ \Γ \vdash e_2 : \σ \\
  \tau \subtype \onerec{} \\
  \sigma \subtype \onerec{} \\
  \dom_\undef(\tau) \wedge \dom_\undef(\sigma) = \varnothing
  }{%
    \Γ \vdash e_1 \orthplus e_2 : \tau \orthplus \sigma
  }
  \irlabel{ROrthMerge}
  \and
  \inferrule{%
    \Γ \vdash e_1 : \tau \\ \Γ \vdash e_2 : \σ \\
  \tau \subtype \onerec{} \\
  \sigma \subtype \onerec{}
  }{%
    \Γ \vdash e_1 + e_2 : \tau + \sigma
  }
  \irlabel{RMerge}
  \and
  \inferrule{%
    \Γ \vdash e_1 : \τ \\ \Γ \vdash e_2 : \bigvee\limits_{i=1}^n s_i
  }{%
    \Γ \vdash e_1 . e_2 : \left( \bigvee\limits_{i=1}^n \τ(s_i) \right) \backslash \undef
  }
  \irlabel{RAccessFinite}
  \and
  \inferrule{%
    \Γ \vdash e_1 : \τ \\ \Γ \vdash e_2 : \σ \\ \σ \subtype \ty{string}
  }{%
    \Γ \vdash e_1 . e_2 : \left( \deff(\τ) \vee \bigvee\limits_{s \in \dom(\τ)} \τ(s) \right) \backslash \undef
  }
  \irlabel{RAccessInfinite}
\end{mathpar}

  \caption{Typing rules for records\label{typing::records}}
\end{figure}

\begin{figure}
  \todo{add rules for operators}
  \caption{Typing rules for builtins operators\label{typing::operators}}
\end{figure}
