\subsection{The $\λ\&-calculus$}

\subsubsection{Patterns}

The typing of patterns is inspired by~\cite{Fri04}: To every pattern $p$, we
associate a type $\Lbag p \Rbag$. The exact definition of $\Lbag \_ \Rbag$ is
given in figure~\pref{typing::pattern-accept}.

The intuition behind this operator is that $\matchType{p}$ represents the type
of the values that will be accepted by the pattern. So for example,
$\matchType{x}$ is equal to $\any$ as the pattern $x$ matches any value.
Likewise, $\matchType{\{ x, y ? 1 \}}$ is $\{ x = \any; y = \any \vee \undef
\}$.

We also define the matching $\ofTypeP{p}{\τ}$ of a type $\τ$ against a pattern
$p$ (in figure~\pref{typing::pattern-ty-match}).
This represents the typing constraints that we can deduce from the applications
of a type $\τ$ to the pattern $p$. It is used to convert type annotations into
concrete typing environments. So for example, $\ofTypeP{\{ x, y \}}{\{ x = \τ;
y = \σ \}}$ is equal to the typing constraints $x : \τ$ and $y : \σ$, which may
be used to type the body of the corresponding function.

The typing rules are given by the
figures~\pref{typing::lambda-calculus},~\pref{typing::records}
and~\pref{typing::operators}.

\begin{figure}
  \begin{center}
    \begin{tabular}{rl}
      \eqdefa{$\matchType{x}$}{$\any$}{}
      \eqdefa{$\matchType{q@x}$}{$\matchType{q}$}{}
      \eqdefa{$\matchType{\{ f_1, \cdots, f_n \}}$}%
        {$\{ \matchType{f_1}; \cdots \matchType{f_n}; \}$}{}
      \eqdefa{$\matchType{l}$}{$l = \any$}{}
      \eqdefa{$\matchType{l ? e}$}{$l = \any \vee \undef$}{}
      \eqdefa{$\matchType{\text{Cons}(x_1, x_2)}$}{$\text{Cons}(\any, \any)$}{}
    \end{tabular}
  \end{center}
  \caption{Semantics of the $\matchType{\_}$ operator%
  \label{typing::pattern-accept}}
\end{figure}
\begin{figure}
  \begin{tabular}{rl}
    \eqdefa{$\ofTypeP{x}{\τ}$}{$\sfrac{x}{\τ}$}{}
    \eqdefa{$\ofTypeP{q@x}{\τ}$}{$\sfrac{x}{\τ}; \ofTypeP{q}{x}$}{}
    \eqdefa{$\ofTypeP{\{\}}{\{ x_1 ? c_1, \cdots, x_n ? c_n \}}$}{%
      $\sfrac{x_1}{\mathcal{B}(c_1)}; \cdots; \sfrac{x_n}{\mathcal{B}(c_n)}$}{}
    \eqdefa{%
      $\ofTypeP{%
        \{ s_1 = \τ_1; \cdots; s_m = \τ_m; \}%
      }{%
        \{x_1 ? c_1, \cdots, x_n ? c_n, \textbf{\ldots}\}%
      }$%
    }{%
      $\sfrac{x_1}{\mathcal{B}(c_1)}; \cdots; \sfrac{x_n}{\mathcal{B}(c_n)}$%
    }{%
      if %
      $\forall (i,j) \in \discrete{1}{m} \times \discrete{1}{n}, s_i \neq s_j$%
    }
    \eqdefa{$\ofTypeP{\{ s = \τ;\}}{\{ x \}}$}{$\sfrac{x}{\τ}$}{if $s = x$}
    \eqdefa{$\ofTypeP{\{ s = \τ;\}}{\{ x ? c \}}$}{$\sfrac{x}{\τ}$}{if $s = x$}
    \eqdefa{$%
      \ofTypeP{\{ s_1 = \τ_1; \cdots; s_n = \τ_n \}}{\{ x, f_1, \cdots, f_m \}}%
    $}{$%
      \sfrac{x}{\τ};%
      \ofTypeP{\{ s_2 = \τ_2; \cdots; s_n = \τ_n \}}{\{ f_1, \cdots, f_m \}}%
    $}{if $s_1 = x$}
    \eqdefa{
      $\ofTypeP{%
        \{ s_1 = \τ_1; \cdots; s_n = \τ_n \}}%
        {\{ x ? c, f_1 \cdots, f_m \}}$%
      }{%
        $\sfrac{x}{\τ};%
        \ofTypeP{\{ s_2 = \τ_2; \cdots; s_n = \τ_n \}}%
          {\{ f_1, \cdots, f_m \}}$%
      }{if $s_1 = x$}
    \eqdefa{%
      $\ofTypeP{\text{Cons}(\τ_1, \τ_2)}{\text{Cons}(x_1, x_2)}$%
    }{$\sfrac{x_1}{\τ_1}; \sfrac{x_2}{\τ_2}$}{}
  \end{tabular}
  \caption{Semantics of $\ofTypeP{p}{\τ}$%
  \label{typing::pattern-ty-match}}
\end{figure}
\begin{figure}
  \begin{mathpar}
  \inferrule{ }{\Gamma; x:\tau \vdash x:\tau}(Var)

  \and
  \inferrule{ }{\Gamma \vdash c:\mathcal{B}(c)}(Const)

  \and
  \inferrule{%
    \Gamma \vdash e_1 : \tau_1 \\ \Gamma \vdash e_2 : \tau_2 \\
    \tau_1 \subtype \zero \rightarrow \one \\
    \tau_2 \subtype \dom(\tau_1)
  }{%
    \Gamma \vdash e_1 e_2 : \tau_1 \tau_2
  }
  (App)

  \and
  \inferrule{%
    \forall i \in \discrete{1}{n}, \\
      \Gamma; \ofTypeP{p}{\sigma_i} \vdash e: \tau_i \\
  }{%
    \lambda \left(p:{\bigvee\limits_{i=1}^n\σ}\right).e :
      \bigwedge\limits_{i=1}^n \left(\σ \rightarrow \τ\right)
  }
  (Abs)

  \and
  \inferrule{%
    \Gamma \vdash e : \tau \\
      \Γ; x : \τ \wedge t \vdash e_1 : \sigma_1 \\
      \Γ; x : \τ \wedge \lnot t \vdash e_2 : \sigma_2
  }{%
    \Gamma \vdash ((x := e \in t) ? e_1 : e_2) : \sigma_1 \vee \sigma_2
  }
  (Tcase)

  \and
  \inferrule{%
    \forall i \in \discrete{1}{n}, \Γ \vdash e_i : \τ_i \\
      \Γ; x_1 : \τ_1; \ldots; x_n : \τ_n \vdash e : \τ
  }{%
    \Gamma \vdash \text{let } x_1 = e_1; \ldots{}; x_n = e_n
      \text{ in } e : \τ
  }
  (Let)

  \and
  \inferrule{%
    \forall i \in \discrete{0}{n},
      \Γ; x_1 : \τ_1; \ldots; x_n : \τ_n \vdash e_i : \τ_i
  }{%
    \Gamma \vdash \text{let rec } x_1 : \τ_1 = e_1; \ldots{}; x_n : \τ_n = e_n
      \text{ in } e_0 : \τ_0
  }
  (LetRec)
\end{mathpar}

  \caption{Typing rules for the $\λ\&-calculus$\label{typing::lambda-calculus}}
\end{figure}
\begin{figure}
    \begin{mathpar}
  \inferrule{%
    \Γ \vdash e' : \τ \\
  \Γ \vdash e : \bigvee\limits_{i=1}^n s_i
  }{%
    \Γ \vdash \left\{ e = e'; \right\} :
  \bigvee\limits_{i=1}^n \left\{ s_i: \τ \right\}
  }
  \irlabel{RFinite}
  \and
  \inferrule{%
    \Γ \vdash e' : \τ \\
  \Γ \vdash e : \σ \\
  \σ \subtype \ty{string}
  }{%
    \Γ \vdash \left\{ e = e'; \right\} :
  \left\{ \_: \τ \vee \undef \right\}
  }
  \irlabel{RInfinite}
  \and
  \inferrule{%
    \Γ \vdash e_1 : \tau \\ \Γ \vdash e_2 : \σ \\
  \tau \subtype \onerec{} \\
  \sigma \subtype \onerec{} \\
  \dom_\undef(\tau) \wedge \dom_\undef(\sigma) = \varnothing
  }{%
    \Γ \vdash e_1 \orthplus e_2 : \tau \orthplus \sigma
  }
  \irlabel{ROrthMerge}
  \and
  \inferrule{%
    \Γ \vdash e_1 : \tau \\ \Γ \vdash e_2 : \σ \\
  \tau \subtype \onerec{} \\
  \sigma \subtype \onerec{}
  }{%
    \Γ \vdash e_1 + e_2 : \tau + \sigma
  }
  \irlabel{RMerge}
  \and
  \inferrule{%
    \Γ \vdash e_1 : \τ \\ \Γ \vdash e_2 : \bigvee\limits_{i=1}^n s_i
  }{%
    \Γ \vdash e_1 . e_2 : \left( \bigvee\limits_{i=1}^n \τ(s_i) \right) \backslash \undef
  }
  \irlabel{RAccessFinite}
  \and
  \inferrule{%
    \Γ \vdash e_1 : \τ \\ \Γ \vdash e_2 : \σ \\ \σ \subtype \ty{string}
  }{%
    \Γ \vdash e_1 . e_2 : \left( \deff(\τ) \vee \bigvee\limits_{s \in \dom(\τ)} \τ(s) \right) \backslash \undef
  }
  \irlabel{RAccessInfinite}
\end{mathpar}

  \caption{Typing rules for records\label{typing::records}}
\end{figure}

\begin{figure}
  \todo{add rules for operators}
  \caption{Typing rules for builtins operators\label{typing::operators}}
\end{figure}
