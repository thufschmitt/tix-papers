\documentclass{article}

\usepackage{fontspec}
\usepackage[usenames,dvipsnames,svgnames,table]{xcolor}
\usepackage{amssymb}

\usepackage{mathtools}
\usepackage{xfrac}

\newcommand{\ty}[1]{\texttt{#1}}
\newcommand{\set}[1]{\ensuremath{\mathcal{#1}}}
\newcommand{\undef}{\oslash}
\newcommand{\quasiconst}{\twoheadrightarrow}
\DeclareMathOperator\dom{dom}

\title{Typing of records in nix}
\author{Théophane Hufschmitt}
\date{}

\begin{document}

\maketitle{}

\section{Static labels}
\subsection{Typing in a CBV setting}
The typing of records in a CBV language is already given (in absence of
polymorphism) by the formalism given in part 4.5 of~\cite{Cas15} (which itself
is the reformulation of the formalism presented by Alain Frisch in its Phd
Thesis − see~\cite{Fri04}).

In the original formalism, records (resp.\ record types) are interpreted as
\emph{quasi-constant} functions from \ty{string} to $\textbf{Values} \cup
\undef$ (resp.\ from \ty{string} to $\textbf{Types} \cup \undef$), where
\begin{itemize}
  \item A \emph{quasi-constant} function from \set{L} to \set{Z} is a
    function $r: \set{L} \rightarrow \set{Z}$ which is constant to an element
    $z \in \set{Z}$, except for a finite set $\dom(r)$. We note $\set{L}
    \quasiconst \set{Z}$ for the set of quasi-constant functions from \set{L}
    to \set{Z}.
  \item \textbf{Values} denotes the set of values in the language.
  \item \textbf{Types} denotes the set of values in the language.
  \item $\undef$ is a distinguished constant that represents an undefined
    field.
\end{itemize}

(in this formalism, the constant $\undef$ was called $\bot$, we renamed it here
in order to avoid confusion with the type $\bot$ representing an undefined
computation).

\subsection{Adaptation to a call-by-name semantic}

As long as we do not allow anything but constant strings as labels, this
formalism requires only a few modifications in order to be useful in a CBN
setting. In fact it is sufficient (I think) to interpret a record as an element
of $\ty{string} \quasiconst \textbf{Expressions} \cup \undef$ (instead of
$\ty{string} \quasiconst \textbf{Values} \cup \undef$).

\section{Dynamic labels}

In nix, the labels can be not only static strings, but also arbitrary
expressions (whose value can't by consequence be statically known in the
general case), which makes the typing more complicated and way less accurate.

One problem raised by this is the semantic of labels evaluation. When the
labels were static, there was no question about their evaluation.

\bibliographystyle{alpha}
\bibliography{../references}
\end{document}


