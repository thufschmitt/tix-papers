\subsection{Nix grammar}
\label{sec:nix-grammar}

The grammar of nix is given at Figures~\pref{grammar::nix}
and~\pref{grammar::types}. It consists\todo{Remove as soon as the grammar is
extended} of a simple lambda calculus with lists (and type annotations).

\begin{figure}
  \begin{grammar}
  \bfseries
  <e> ::=
    $x$ \| $c$
    \alt $\λ$$p$.$e$ \| $e$ $e$
    \alt let $x$ = $e$; $\cdots{}$; $x$ = $e$; in $e$
    \alt [ e $\cdots$ $e$ ]
    \alt if $e$ then $e$ else $e$

  <c> ::= $s$ \| $i$ \| $b$

  <p> ::= $q$ \| $q$@$x$ \| $r$

  <q> ::= ($r$, $r$) \| $q$:\τ

  <r> ::= $x$ | $x$:\τ
\end{grammar}

  \caption{\label{grammar::nix}The nix grammar for expressions}
\end{figure}

\begin{figure}
  \begin{grammar}
  \bfseries
  <t> ::= \c/ \| \t/ $\bm{\rightarrow}$ \t/
    \alt \t/ $\bm{\vee}$ \t/ \| \t/ $\bm{\wedge}$ \t/ \| \t/ $\bm{\backslash}$ \t/
    \alt [\meta{R}]
    \alt \{ \s/ = \u/; \ldots{}; \s/ = \u/; _ = \u/ \}

  <u> ::= \t/ \| ?\t/

  <R> ::= \t/ \| \meta{R^{\bm{+}}} \| \meta{R}?
    \| \meta{R} $\bm{\vee}$ \meta{R} \| \meta{R} $\bm{\wedge}$ \meta{R}

  <\τ> ::= \t/ % No polymorphism for now
\end{grammar}

We write \textbf{\{ \s/ = \u/; \ldots{}; \s/ = \u/; \}} as syntactic
sugar for \textbf{\{ \s/ = \u/; \ldots{}; \s/ = \u/; _ = $\undef$ \}}, and
\textbf{\{ \s/ = \u/; \ldots{}; \s/ = \u/; .. \}} as syntactic sugar for
\textbf{\{ \s/ = \u/; \ldots{}; \s/ = \u/; _ = \textmd{\emph{Any}} \}})

  \caption{\label{grammar::types}The nix and nix-light grammar for types}
\end{figure}

\subsection{nix-light}
\label{sec:nix-light-grammar}

The grammar of nix-light is based of the grammar of nix and brings several
modifications:
\begin{itemize}
  \item A first huge change is the removal of the \emph{if} construct which is
    replaced by a more general \emph{typecase} which is easier to reason on.

  \item Another notable change is that the opaque list construct of nix is
    replaced by the classical \texttt{nil} and \texttt{cons}.
    This avoids having over-complicated typing and evaluation rules for lists.

    For consistency, a pattern for lists has also been added.
\end{itemize}

The grammar of nix-light is given in the
figures~\pref{grammar::expressions},~\pref{grammar::values}
and~\pref{grammar::types} (the types are the same as nix types).

\begin{figure}
  \begin{grammar}
  \bfseries
  <e> ::=
    $x$ \| $c$
    \alt $e$.$a$ \| $e$.$a$ or $e$
    \alt $\λ$$p$.$e$ \| $e$ $e$
    \alt let $x$ = $e$; $\cdots{}$; $x$ = $e$; in $e$
    \alt let $x$ : $\τ$ = $e$; $\cdots{}$; $x$ : $\τ$ = $e$; in $e$
    \alt Cons ($e$, $e$)
    \alt ($x$ = $e$ $\bm{\in}$ $t$) ? $e$ : $e$

  <c> ::= $s$ \| $i$ \| $b$ \| Nil

  <p> ::= $q$ \| $q$@$x$ \| $r$

  <q> ::= Cons ($r$, $r$) \| Nil \| $q$:\τ

  <r> ::= $x$ | $x$:\τ
\end{grammar}

  \caption{\label{grammar::expressions}The nix-light grammar for expressions}
\end{figure}

\begin{figure}
  \begin{grammar}
  \bfseries

  <v> ::=
    $c$
    \alt $\λ p$.$e$
    \alt Cons ($e$, $e$) \| Nil
\end{grammar}

  \caption{\label{grammar::values}The nix-light grammar for values}
\end{figure}
